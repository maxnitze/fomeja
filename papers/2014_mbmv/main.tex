\documentclass[a4paper,fontsize=12pt,headings=small,captions=tableheading,%
numbers=endperiod,abstracton,pdftex]{scrartcl}

%!TEX root = main.tex
\usepackage[utf8]{inputenc}
\usepackage[T1]{fontenc}

\usepackage{listings}
\lstset{
  basicstyle=\ttfamily\scriptsize,
  breaklines
}

% Notes
\usepackage{manfnt}
\def\hang{\hangindent19pt}
\def\d@anger{\medbreak\begingroup\clubpenalty=10000
 \def\par{\endgraf\endgroup\medbreak} \noindent\hang\hangafter=-2
 \hbox to0pt{\hskip-\hangindent\dbend\hfill}\small}
\outer\def\danger{\d@anger}

% inline references
\usepackage{prettyref}
\newrefformat{cha} {Chapter~\ref{#1}}
\newrefformat{sec} {Section~\ref{#1}}
\newrefformat{fig} {Fig.~\ref{#1}}
\newrefformat{tab} {Table~\ref{#1}} 
\newrefformat{lst} {Listing~\ref{#1}} 
\newrefformat{eq}  {Eq.~\ref{#1}}
\newrefformat{ex}  {Example~\ref{#1}}

%%% Local Variables:
%%% mode: latex
%%% TeX-master: "main"
%%% End:


\title{Formale Methoden für Alle} \subtitle{(Erweiterte Zusammenfassung)}
\author{%
  \authorblock{%
    Mathias Soeken$^{\rm 1,2}$ \qquad Max Nitze$^{\rm 1}$ \qquad Rolf Drechsler$^{\rm 1,2}$ \\
    $^{\rm 1}$Arbeitsgruppe Rechnerarchitektur, Universität Bremen \\
    $^{\rm 2}$Cyber-Physical Systems, DFKI GmbH, Bremen \\
    \url{{msoeken,maxnitze,drechsle}@informatik.uni-bremen.de}}
}

\begin{document}

\maketitle

\vspace{-1.5cm}

\begin{abstract}
  Formale Methoden können verwendet werden, um schwierige Probleme exakt zu
  lösen.  Ihre Anwendung ist jedoch nicht trivial und erfordert in der Regel
  Expertenwissen.  Wir stellen das Werkzeug \textsc{DIAB} vor, das formale
  Methoden näher an Programmierer bringt, mit dem Ziel die Komplexität der
  Verwendung formaler Methoden an die der Implementierung von Programmen
  anzunähern.  Konkret werden in dieser Arbeit Erfüllbarkeitsbeweiser
  verwendet, um algorithmische Probleme automatisch aus einer Programmiersprache
  heraus zu lösen.  Die Transformation in das Erfüllbarkeitsproblem übernimmt
  dabei das Werkzeug.
\end{abstract}

\begin{figure}[t!]
\begin{minted}[fontsize=\fontsize{10pt}{10pt}\selectfont,bgcolor=bg]{java}
import DIAB;
...

public class Vertex {
  private int id;

  @Variable
  private int color;
  public int getColor() { return color; }
  ...
}

public class Edge {
  private Vertex first, second;

  public Vertex getFirst() { return first; }
  public Vertex getSecond() { return second; }
  ...
}

public class Graph {
  private Collection<Vertex> vertices;
  private Collection<Edge> edges;
  private int k;
  ...

  public void parse(final String filename) { ... }

  @Constraint
  public boolean maximumColor(final Vertex vertex) {
    return vertex.color >= 0 && vertex.color < k;
  }

  @Constraint
  public boolean differentColor(final Edge edge) {
    return edge.getFirst().getColor() != edge.getSecond().getColor();
  }

  public static void main(String[] args) {
    Graph g = new Graph();
    g.parse("some_graph.col");
    DIAB.satisfy(g);
  }
}
\end{minted}
\medskip
\caption{Illustration für die direkte Einbettung formaler Methoden zum Finden
  einer Färbung}
\label{fig:diab-idea}
\end{figure}

\section{Einleitung und Motivation}
% Für die Entwicklung eines einfachen Computerprogramms war vor einem halben
% Jahrhundert ein sehr spezielles Expertenwissen unumgänglich.  Beispielsweise war
% es üblich in den frühen Ausgaben der \textsl{Communications of the ACM}
% vollständigen Programmcode abzudrucken~(vgl.~beispielsweise~\cite{Gib:75}).
% Durch die Entwicklung von Sprachen wie \textsc{Pascal} oder \textsc{C} Anfang
% der 70er Jahre wurde die Disziplin zunächst einem breiteren Publikum zugänglich
% gemacht.  Heutzutage ist die Entwicklung von Computerprogrammen für viele
% durch
% \begin{inparaenum}[(i)]
% \item Integration in den Schulunterricht,
% \item zahlreiche domänenorientierte Sprachen für dedizierte Probleme, und
% \item soziale Netzwerke zum Thema Programmierung wie
%   GitHub,\footnote{github.org} Stack Overflow\footnote{stackoverflow.com} oder
%   Code School\footnote{codeschool.com}
% \end{inparaenum}
% einfach möglich.  All dies erlaubt steile Lernkurven, so dass sich mit wenig
% Aufwand Webseiten, Apps für Smart Phones oder Mikrocontroller auch durch
% Nicht-Experten programmieren lassen.

Formale Methoden befinden sich heute in etwa dem Zustand in dem sich
Programmiersprachen vor 50~Jahren befanden.  Mithilfe von formalen Methoden
lassen sich zwar viele schwierige Probleme lösen, vorhandene Werkzeuge sind
jedoch nicht trivial und ihre Anwendung benötigt insbesondere auch ein gutes
Verständnis der zugrundeliegenden Algorithmen.  Die Eigenschaft, dass formale
Methoden in externen Tools mit eigenen neuen Programmiersprachen ausgelagert
sind, erschwert ihre Verwendung zusätzlich.

In dieser erweiterten Zusammenfassung wird ein Ansatz illustriert, der
Programmierern formale Methoden zugänglich macht, ohne dass sie direkt sichtbar
sind.  Darüberhinaus ist der Algorithmus in eine existierende Programmiersprache
eingebettet, in diesem Fall \textsc{Java}, so dass die gewohnte Arbeitsumgebung
nicht verlassen werden muss.

Das Problem und die Lösungsidee sollen im Folgenden am Beispiel von
Graphenfärben verdeutlicht werden, der Ansatz lässt sich aber analog auch auf
andere Probleme anwenden.  Ein Graph ist genau dann \emph{$k$-färbbar}, wenn
alle Knoten mit einer von $k$ Farben gefärbt sind und benachbarten Knoten
unterschiedliche Farben zugeordnet sind.  Es soll ein Programm entwickelt
werden, dass einen Graphen ohne Färbung einliest und anschließend den Graphen
mit~$k$ Farben färbt, falls eine solche Färbung existiert.  Viele praktisch
relevante Probleme lassen sich auf das Färben von Graphen
abbilden~(z.B.~Registerallokation~\cite{CAC+:81}).

Eine naive Umsetzung für dieses NP-vollständige Problem~\cite{Kar:72} könnte
alle Färbungen aufzählen bis eine valide gefunden worden ist.  Aufwendigere
Implementierungen würden eine \emph{branch and bound} Variante mit
\emph{backtracking} verwenden.

Mithilfe des von uns entwickelten Ansatzes \textsc{DIAB} lässt sich das Problem
wie in Abbildung~\ref{fig:diab-idea} umsetzen.  Der Graph verwendet zwei
Datenstrukturen für Knoten und Kanten, wobei der Knoten eine Feldvariable
\mintinline{java}|color| enthält, die durch die Annotation
\mintinline{java}|@Variable| gekennzeichnet ist.  Die \mintinline{java}|Graph|
Klasse hat eine Methode zum Parsen von Dateien.  Die anderen Methoden sind als
\mintinline{java}|@Constraint| annotiert und beschreiben die Bedingungen für
eine valide Graphfärbung.  Dies ist eine direkte Übersetzung der Definition für
eine Graphfärbung in \textsc{Java} Quellcode.  Der Code ist ausreichend für die
Implementierung (alle durch `\mintinline{java}|...|' ausgelassenen Zeilen
beschreiben Methoden zum Aufbauen des Graphen).  In der \mintinline{java}|main|
Methode wird eine Instanz des Graphen erzeugt und eine Datei eingelesen. Danach
ist der Graph zunächst ungefärbt.  Der Aufruf von
\mintinline{java}|DIAB.satisfy(g)| belegt alle mit \mintinline{java}|@Variable|
annotierten Feldvariablen, die über die Graphinstanz erreicht werden können, so
dass alle Bedingungen der Methoden, die als \mintinline{java}|@Constraint|
annotiert sind, erfüllt werden.

Ein ähnlicher Ansatz ist im Tool Pex~\cite{TS:05} implementiert, der formale
Methoden für \emph{white box testing} in~.NET Sprachen nutzt.  Die
Programmiersprache \textsc{Eiffel}~\cite{Meyer:88} bindet durch das
\emph{Design-by-contract} Paradigma formale Methoden auf Syntaxebene an die
Sprache.

\section{Implementierung}
In diesem Abschnitt soll die Implementierung des Algorithmus illustriert werden.
Zunächst wird beschrieben wie durch \mintinline{java}{@Constraint} annotierte
Methoden, im Folgenden \emph{Constraint Methoden} genannt, übersetzt werden, so
dass formale Methoden angewandt werden können.  Im Anschluss werden die aktuell
implementierten und geplanten Features diskutiert.

\subsection{Übersetzung der Constraint Methoden}
Die Implementierung der \mintinline{java}|satisfy| Methode, wie sie in
Abbildung~\ref{fig:diab-idea} angewendet wird, ist abstrakt wie folgt
realisiert.  Zunächst werden alle annotierten Constraint Methoden mithilfe der
\textsc{Java} \emph{Reflection API} ermittelt.  Der Bezug der Bedingungen ergibt
sich aus den Parametern der Methoden.  In der Constraint Methode
\mintinline{java}|maximumColor| muss die Bedingung für alle Knoten gelten, die
über den Graphen erreicht werden können.  Diese können direkt über die
Feldvariable \mintinline{java}|vertices| und indirekt über die Feldvariable
\mintinline{java}|edges| erreicht werden.  Da dies per Reflection API erfolgt,
muss satisfy keine Eigenschaften der Datenstruktur kennen.  Die Constraint
Methoden werden zur Laufzeit dis\-as\-sem\-bliert und automatisch über den
\textsc{Java} Bytecode in prädikatenlogische Formeln erster Ordnung übersetzt.
Die freien Variablen werden über die \mintinline{java}|@Variable| Annotation
ermittelt.  Alle Formeln werden in einer Konjunktion zusammengeführt und das
resultierende Erfüllbarkeitsproblem wird mit einem externen SMT Beweiser, im
konkreten Fall Z3~\cite{MB:08}, gelöst.  Aus einer möglichen erfüllenden
Belegung können dann die konkreten Werte für die mit
\mintinline{java}|@Variable| annotierten Feldvariablen ausgelesen und zugewiesen
werden.

\subsection{Aktueller Stand und nächste Schritte}
Unsere Implementierung unterstützt aktuell Programme im Stil des Programms aus
Abbildung~\ref{fig:diab-idea}.  Das zu lösende Problem muss ein
Erfüllbarkeitsproblem sein.  Es ist auch möglich externe Bibliotheken in der
Implementierung zu verwenden.  Beispielsweise haben wir das
Graphenfärbenbeispiel auf Basis der JUNG\footnote{jung.sourceforge.net}
Bibliothek implementiert, anstatt eine eigene Graphklasse zu definieren.  Als
Datentypen für freie Variablen werden derzeit \mintinline{java}|int| und
\mintinline{java}|boolean| unterstützt.  Es ist möglich lokale Variablen in den
Constraint Methoden zu verwenden, so lange sie \mintinline{java}|final|
deklariert sind, d.h.~ihr Wert lässt sich nach initialer Zuweisung nicht mehr
verändern.  Operationsaufrufe in Constraint Methoden sind möglich, solange die
aufrufenden Methoden keine Seiteneffekte haben.

\begin{figure}[t]
  \centering
  \begin{minted}[fontsize=\fontsize{10pt}{10pt}\selectfont,bgcolor=bg]{java}
public class Graph {
  ...

  @Objective
  public int numColors() {
    final Set<int> colors = new HashSet<int>();
    for (Vertex v : vertices) {
      colors.add(v.getColor());
    }
    return colors.size();
  }

  public static void main(String[] args) {
    Graph g = new Graph();
    g.parse("some_graph.col");
    DIAB.minimize(g);
  }
}
  \end{minted}
  \medskip
  \caption{Unterstützung von Optimierungsproblemen}
  \label{fig:optimization}
\end{figure}

In den nächsten Schritten sollen weitere Basisdatentypen wie beispielsweise
\mintinline{java}|float| unterstützt werden.  Außerdem soll untersucht werden,
welche Vorteile funktionale JVM Sprachen wie \textsc{clojure} oder
\textsc{Scala} bei der Übersetzung in prädikatenlogische Formeln haben.  Formale
Methoden sind nicht nur zum Lösen von Erfüllbarkeitsproblemen hilfreich, sondern
auch für Optimierungsprobleme.  Dies soll auch untersucht werden.  Denkbar sind
Programmkonstrukte wie in Abbildung~\ref{fig:optimization} illustriert.  Hier
wird die Methode \mintinline{java}|numColors| als Optimierungskriterium durch
die Annotation \mintinline{java}|@Objective| gekennzeichnet.  In der
\mintinline{java}|main| Methode wird anstatt \mintinline{java}|satisfy| die
Funktion \mintinline{java}|minimize| aufgerufen, um die Anzahl der Farben im
gefärbten Graph zu minimieren.


\section{Zusammenfassung}
Wir haben den Ansatz \textsc{DIAB} illustriert, der formale Methoden näher an
Programmierer bringt.  Anstatt wie bisher auf externe Tools mit eigenen
dedizierten Sprachen angewiesen zu sein, ist der vorgestellte Ansatz direkt in
die Programmiersprache \textsc{Java} eingebettet.  Dabei sind die Algorithmen
als Klassenbibliothek implementiert worden ohne die Sprache ändern zu müssen.

\bibliography{literatur}

\end{document}


%%% Local Variables:
%%% mode: latex
%%% TeX-master: t
%%% mode: fci
%%% mode: reftex
%%% mode: whitespace
%%% mode: auto-fill
%%% mode: flyspell
%%% eval: (ispell-change-dictionary "deutsch8")
%%% End:
